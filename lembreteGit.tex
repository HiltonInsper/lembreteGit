\documentclass[10pt,a4paper]{article}

%% Language and font encodings
\usepackage[brazilian]{babel}
\usepackage[utf8x]{inputenc}
\usepackage[T1]{fontenc}

%% Sets page size and margins
\usepackage[a4paper,top=2cm,bottom=1.5cm,left=2.5cm,right=2.5cm,marginparwidth=1.5cm]{geometry}

%% Useful packages
\usepackage{amsmath}
\usepackage{amsfonts}
\usepackage{amssymb}
\usepackage{graphicx}
\usepackage[colorinlistoftodos]{todonotes}
\usepackage[colorlinks=true, allcolors=blue]{hyperref}

\title{Lembrete para uso do {\tt git}}
% \author{You}

\begin{document}
\maketitle

% \begin{abstract}
% Your abstract.
% \end{abstract}

\noindent
O texto a seguir é um breve lembrete das diversas etapas necessárias para se ter nossos projetos no {\tt git}, baseados no GitHub. Dúvidas de conteúdo devem ser resolvidas pela consulta ao texto ``A'', que detalha os conceitos e a operação do {\tt git} em mais detalhe e profundidade.

\begin{enumerate}
	\item {\bf Cadastro}

		Todos alunos do grupo devem se cadastrar no GitHub, em \\
		\url{https://github.com/}

	\item {\bf Criação do repositório}

		Apenas um dos alunos do grupo deve criar um repositório para o trabalho em grupo, em \\
		\url{https://github.com/new}

		Não se deve esquecer de criar o conteúdo inicial, selecionando a opção ``{\tt Initialize this repository with a README}''. 
                
	\item {\bf Convite para o grupo}

		O criador do repositório deve convidar todos parceiros do grupo para que eles possam contribuir para o projeto. 
		
		Procure a guia {\tt Settings} na página do projeto. E, dentro dela, escolha a opção  {\tt Collaborators}.

		Eles serão notificados do projeto em sua página no GitHub, e devem aceitar o convite para que façam parte do projeto;
                
	\item {\bf Instalação local do {\tt git}}
	
		O {\tt git} para MS-Windows pode ser baixado em \\
		\url{https://git-scm.com/download/win}

		Para MacOS, use a URL\\
        \url{https://git-scm.com/download/mac}

		Para as diversas distribuições de GNU/Linux, Solaris e BSD, siga as instruções da página {\em Download for Linux and Unix}, em \\
        \url{https://git-scm.com/download/linux}

                
        \item {\bf Configuração local do {\tt git}}
        
		Para configurar o {\tt git} em seu computador, inclua os comandos que vão identificá-lo nos repositórios. 
		
		Um dos locais mais padronizados para se fazer isso é o terminal de comandos de seu computador. No MS-Windows, ele é o {\tt cmd.exe}.  No MacOS, nas distribuições GNU/Linux ele é o {\tt shell}. Em qualquer caso, digite 
		\begin{quote}
		\begin{tt}		
			git config --global user.name "Seu Nome"
		\end{tt}			
		\end{quote} 
		onde {\tt {\em Seu Nome}} deve ser trocado pelo nome do usuário, e depois digite
		\begin{quote}
		\begin{tt}		
			git config --global user.email "seu.email@mailserver.com"
		\end{tt}			
		\end{quote} 
	
                
	\item {\bf Clonagem do repositório do \url{github.com}}
    	Para criar uma cópia de seu repositório, hospedado no \url{github.com}, clique no botão {\tt Clone or download} na aba {\tt Code} da página de seu projeto. Isto fará surgir uma pequena janela. Clique no botão que está à direita da caixa de texto que tem a URL iniciada por {\tt https://github.com}
    	
    	Crie um diretório que vai hospedar seu projeto {\tt git}. Por exemplo, em sua pasta pessoal (que é mostrada quando se entra no terminal de comandos), crie o diretório {\tt git}. 
    	
    	Então vá ao terminal de comandos (ou {\em shell}) comentado no item anterior e digite 
		\begin{quote}
		\begin{tt}		
			git clone
		\end{tt}			
		\end{quote} 
		e, sem pressionar {\tt Enter}, cole no terminal a URL copiada do portal \url{github.com}
		
		Finalmente pressione {\tt Enter}. Isto vai transferir os arquivos de seu repositório no \url{github.com} para o repositório local, em seu computador.
		
		Agora você pode trabalhar com o repositório local, manter um histórico de suas alterações dos arquivos e, finalmente, transferir arquivos do \url{github.com} para sua máquina e de sua máquina para o \url{github.com}, através do comando {\tt git}.
                
	\item {\bf Criação de novos arquivos}

    	Os novos arquivos de seu projeto devem todos ser criados no diretório que o {\tt git} acabou de criar. Recapitulando, se você chamou o diretório para criar repositórios com o nome de {\tt git}, como sugerido, e se seu projeto no \url{github.com} se chama {\tt Proj1}, a operação {\tt clone} criou o diretório {\tt Proj1} dentro do diretório {\tt git}.
    	
    	Crie seus arquivos de programa normalmente dentro deste diretório, através de seu editor preferido, que o {\tt git} se encarregará de acompanhar suas versões;
                
    \item {\bf Ciclo de trabalho}
        
	\item {\bf Transferência para repositório local}
                
    \item {\bf Transferência para repositório no \url{github.com}}
        
\end{enumerate}

\end{document}